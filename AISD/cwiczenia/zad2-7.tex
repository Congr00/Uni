\documentclass[11pt, wide]{article}

    \usepackage[polish]{babel}
    \usepackage{polski}
    \usepackage{secdot}
    \frenchspacing
    \usepackage[utf8x]{inputenc}
    \usepackage[T1]{fontenc}
    
    \usepackage{amsfonts}
    \usepackage{graphicx}
    \usepackage[colorinlistoftodos]{todonotes}
    \usepackage[colorlinks=true, allcolors=blue]{hyperref}
    
    \usepackage[shortlabels]{enumitem}
    \usepackage[]{algorithm2e}
    \usepackage{mathtools}
    \DeclarePairedDelimiter\ceil{\lceil}{\rceil}
    \DeclarePairedDelimiter\floor{\lfloor}{\rfloor}
    
    \setlength{\parindent}{0pt}
    \setlength{\parskip}{1ex plus 0.5ex minus 0.2ex}
    \newcommand\doubleplus{+\kern-1.3ex+\kern0.8ex}
    
    \title{Rozwiązanie zadania 2.7}
    \author{Łukasz Klasiński}
    
    \begin{document}
    \maketitle
    
    \tableofcontents
    
    \section{Problem}
    
    System złożony z dwóch maszyn A i B wykonuje n zadań. Każde z zadań wykonywane jest
    na obydwu maszynach, przy czym wykonanie zadania na maszynie B można rozpocząć dopiero
    po zakończeniu wykonywania go na maszynie A. Dla każdego zadania określone są dwie liczby
    naturalne $a_i$ i $b_i$ określające czas wykonania i-tego zadania na maszynie A oraz B (odpowiednio).
    Ułóż algorytm ustawiający zadania w kolejności minimalizującej czas zakończenia wykonania
    ostatniego zadania praz maszynę B.
    
    \section{Algorytm}
    
    \subsection{Idea}
    
    W rozwiązaniu zadania skorzystamy z poniższych lematów i obserwacji.


    \textbf{Lemat 1.}

    Każde rozwiązanie optymalne da się przekształcić tak, aby kolejność
    wykonywania zadań na maszynie A i B była taka sama.
    \\


    \textbf{Lemat 2.}

    Niech 
    \begin{align*}
    S_1 &= \{ a_1, \ldots, a_n, b_1, \ldots, b_n \} \\
    S_2 &= \{ a_1 - d, \ldots, a_n - d, b_1 - d, \ldots, b_n - d \}   
    \end{align*}
    gdzie $d \leq min(S1) $
    oraz $R_1$, $R_2$ będą optymalnymi rozwiązaniami dla kolejno $S_1$ i $S_2$.
    Wtedy użycie uporządkowania $R_1|R_2$ na zbiorze $S_2|S_1$ nie pogorszy czasu optymalnego rozwiązania.
    \\

    \textbf{Lemat 3a.}

    Jeśli czas trwania i-tego zadania $a_i = 0, a_i \in A$, to można przenieść je wraz z 
    odpowiadającym mu zadaniem wykonywanym na maszynie B na początek kolejki bez pogorszenia całkowitego czasu wykonywania zadań.
    
    
    \textbf{Lemat 3b.}

    Jeśli czas trwania i-tego zadania $b_i = 0, b_i \in B$, to można przenieść je wraz z 
    odpowiadającym my zadaniem wykonywanym na maszynie A na koniec kolejki bez pogorszenia całkowitego czasu wykonywania zadań.
    \\

    \textbf{Obserwacja 1.}
    Zauważmy, że z lematu 2 oraz 3a i 3b, natychmiast wnioskujemy, że w rozwiązaniu optymalnym przestawienie 
    najmniejszego elementu, zależnie czy jest to zadanie maszyny A|B kolejno na początek|koniec rozwiązania, zachowuje
    optymalny czas wykonania zadań.
    \\

    \textbf{Obserwacja 2.}
    Załóżmy, że w zbiorze $S$ mamy pary zadań $a_i$, $b_i$, a zbiór $O$ jest pusty. Wyjmijmy ze zbioru $S$ parę z największym elementem.
    Dodajmy ją do zbioru $O$ i usuńmy z $S$. Następnie przesuńmy na początek|koniec listy, zależnie czy wybraliśmy element z pary, należący do $A|B$.
    Oczywiście po dodaniu pierwszego elementu $O$ wykonuje się optymalnie. Powtarzajmy, aż zbiór $S$ będzie pusty. 
    Tak budowany zbiór $O$ po dodaniu kolejnego elementu, przesunie go na początek|koniec listy,
    ,a ponieważ będzie on najmniejszy to z obserwacji 1, wiemy, że takie przesunięcie zachowuje optymalny czas wykonania zadań. 
    Zatem po opróżnieniu zbioru $S$ zbiór $O$ będzie optymalny.

    \subsection{Pseudokod}

    Zauważmy, że zbiór $O$ z obserwacji 2, jest do pewnego momentu posortowany
    rosnąco po elemencie $a_i \in A$ z pary $(a_i, b_i)$, a potem malejąco względem
    elementu $b_i$. Zatem możemy skonstruować algorytm, który będzie porządkować te 
    pary zależnie od tego który element był mniejszy w dwóch oddzielnych zbiorach $L, R$,
    które po wykorzystaniu wszystkich zadań będą połączone i utworzą $O$.


    \begin{algorithm}[H]
     \KwData{$S$; $S = \{ a_1, \ldots, a_n, b_1, \ldots, b_n \}$}
     \KwResult{
         Lista zawierająca kolejne indeksy zadań do wykonania\newline
         na maszynie A i B
      }
     $L,R = []$\;
     \While{$S$ niepuste}{
        $m := min(S)$ \;
        $i := index(m)$\;        
        \If{$m \in A$}{
            $L.push\_back(i)$\;
        }
        \Else{
            $R.push\_front(i)$\;            
        }
        $S = S \setminus a_i$\;
        $S = S \setminus b_i$\;
     }
     \Return concat(L, R)
    \end{algorithm}
    
    \section{Dowód}
    
    \textbf{D-d lematu 1.}

    Weżmy dowolne rozwiązanie optymalne $O$ i należący do niego 
    element $a_i$ taki, że $a_i \in A$. Niech $b_k \in B$ będzie pierwszym elementem, który rozpoczyna się po $a_i$. 
    Wszystkie zadania od $b_k$ włącznie do $b_i$ nazwijmy sekcją. 
    Wyjmijmy z rozwiązania $b_i$. Teraz możemy przesunąć sekcję o czas wykonania $b_i$, ponieważ przesunięcie ich w prawo
    nie zmieni czasu rozwiązania. Utworzyła się przerwa wielkości $b_i$ po $a_i$, gdzie wstawiamy usunięty wcześniej element.
    
    \hspace*{12cm} $\Box$
    \\
    
    \textbf{D-d lematu 2.}

    Weźmy $S_1$,$S_2$ z lematu. Niech $R_1$ będzie optymalnym
    rozwiązaniem dla $S_1$, a $R_2$ dla $S_2$. Przejście z $S_1$ do $S_2$ nazwijmy
    \textbf{\textit{kurczeniem}}, a z $S_2$ do $S_1$ \textbf{\textit{rozciąganiem}}. Pokażemy, że
    $R_1|R_2$ jest optymalne dla dla $S_2|S_1$.     
    \\

    \textbf{Obserwacja \textit{rozciąganie, kurczenie}}

    Podczas \textit{rozciągania} powiększamy elementy z A oraz B
    o jakąś stałą $d \leq min(S_1)$. Ponieważ powiększamy $n$ elementów z A i B, a ostatni element z B 
    określa, kiedy zakończy się czas wykonania zadań, to po rozszerzeniu ostatniego elementu o $d$ w najgorszym 
    przypadku czas wydłuży się o co najwyżej $d*(n+1)$.

    Podobnie w przypadku \textit{kurczenia} czas skróci się o co najwyżej $d*(n+1)$.
    \\

    Załóżmy nie wprost, że $\exists R_1$, że \
    $$T(S_2, R_2) < T(S_2, R_1)$$ oraz $$T(S_1, R_1) < T(S_2, R_1)$$ gdzie $T(S,R)$ 
    oznacza czas wykonania zadań $S$ w kolejności $R$.

    Wtedy z \textit{rozciągania} mamy: $$T(S_2, R_2) + (n+1)d \geq T(S_1, R_2) $$

    natomiast z \textit{kurczenia}: $$T(S_1, R_1) - (n+1)d \geq T(S_2, R_1)$$
    Po kilku prostych przekształceniach dochodzimy do sprzeczności:
    \begin{gather*}
        T(S_2, R_1) \leq T(S_1, R_1) - (n+1)d \\
        T(S_2, R_2) \geq T(S_1, R_2) + (n+1)d \\
        T(S_1, R_1) - (n+1)d \geq T(S_2, R_2) > T(S_2, R_2) \geq T(S1, R2) - (n+1)d \\
        T(S_1, R_2) - (n+1)d > T(S_1, R_2) - (n+1)d \\
        T(S_1, R_2) > T(S_1, R_2) \\
        \bot
    \end{gather*}

    \textbf{D-d lematu 3a.}
    Zadanie $a_i$, możemy przestawić na początek kolejki maszyny $A$, ponieważ nic nie trwa, więc nie
    przesunie pozostałych zadań. Natomiast odpowiadające zadanie $b_i$
    podobnie jak w dowodzie lematu 1, możemy wyciągnąć, przesunąć poprzedzającą go sekcję o jego czas trwania
    i w wstawić na początek kolejki maszyny $B$ bez pogorszenia całkowitego czasu działania.
    \hspace*{12cm} $\Box$

    \textbf{D-d lematu 3b.}
    Podobnie jak w lemacie 3b, zadanie $b_i$ możemy prestawić na koniec, ponieważ nic nie trwa, więc nie wydłuży czasu wykonania,
    natomiast odpowiadające mu zadanie $a_i$ możemy wyciągnąć, następujące po nim zadania przesunąć w lewo o jego czas trwania i 
    wstawić na koniec kolejki $A$.
    \hspace*{12cm} $\Box$

    \end{document}