\documentclass[a4paper]{article}

%% Language and font encodings
\usepackage[polish]{babel}
\usepackage{polski}
\usepackage{secdot}
\frenchspacing
\usepackage[utf8x]{inputenc}
\usepackage[T1]{fontenc}

%% Sets page size and margins
\usepackage[a4paper,top=3cm,bottom=2cm,left=3cm,right=3cm,marginparwidth=1.75cm]{geometry}

%% Useful packages
\usepackage{amsfonts}
\usepackage{graphicx}
\usepackage[colorinlistoftodos]{todonotes}
\usepackage[colorlinks=true, allcolors=blue]{hyperref}

\usepackage[shortlabels]{enumitem}
\usepackage[]{algorithm2e}
\usepackage{mathtools}
\DeclarePairedDelimiter\ceil{\lceil}{\rceil}
\DeclarePairedDelimiter\floor{\lfloor}{\rfloor}

\setlength{\parindent}{0pt}
\setlength{\parskip}{1ex plus 0.5ex minus 0.2ex}
\newcommand\doubleplus{+\kern-1.3ex+\kern0.8ex}

\title{Rozwiązanie zadania 2.3}
\author{Piotr Pietrzak}

\begin{document}
\maketitle

\tableofcontents

\section{Problem}

Rozważ następującą wersję problemu wydawania reszty: dla danych liczb naturalnych $a$, $b$ $(a \leq b)$ chcemy przedstawić ułamek $\frac{a}{b}$ jako sumę różnych ułamków o licznikach równych 1.

\begin{enumerate}[a)]
\item Udowodnij, że algorytm zachłanny zawsze daje rozwiązanie.
\item Czy zawsze jest to rozwiązanie optymalne (tj. o najmniejszej liczbie składników)?
\end{enumerate}

Najpierw, dla przypomnienia, wprowadzę algorytm, później odpowiem na pytania w zadaniu.

\section{Algorytm}

\subsection{Idea}

W zadaniu rozważamy algorytm zachłanny, czyli taki, który w każdym kroku dokonuje lokalnie najbardziej obiecującego wyboru. Dla naszego problemu, jest to algorytm, który w i-tym kroku bierze największy ułamek $\frac{1}{d_{i}}$, taki że $\frac{1}{d_{i}} < \frac{a_{i}}{b_{i}}$.

Zauważmy, że największym ułamkiem o liczniku $1$ nie większym od $\frac{x}{y}$ jest $\frac{1}{\ceil{\frac{y}{x}}}$. Pozostaje nam wyprowadzić wzór na pozostały ułamek, który wykorzystamy w algorytmie.
$$
	\frac{a_{i}}{b_{i}} = \frac{x}{y}
$$

$$
	\frac{a_{i + 1}}{b_{i + 1}} =
    \frac{x}{y} - \frac{1}{\ceil{\frac{y}{x}}} = \frac{
    	x\ceil{\frac{y}{x}} - y
    }{
    	y\ceil{\frac{y}{x}}
    }
$$

\subsection{Pseudokod}

Rozwiązanie możemy opisać poniższym pseudokodem.


\begin{algorithm}[H]
 \KwData{$a$, $b$; $a \leq b$}
 \KwResult{
 	List of denominators $D = [d_{1}, ..., d_{n}]$,\newline 
	where $\sum_{k=1}^{n} \frac{1}{d_k} = \frac{a}{b}$\newline
    and $\forall d_{i}, d_{j} \in D \quad i \neq j \implies d_{i} \neq d_{j}$
 }
 $D = []$\;
 \While{$a > 0$}{
 	$d = \ceil{\frac{b}{a}}$\;
    $D = [d] \doubleplus D$\;
    $a = a*d - b$\;
    $b = b*d$\;
 }
 \Return D
\end{algorithm}

\section{Dowód}

\begin{enumerate}[a)]
\item Udowodnij, że algorytm zachłanny zawsze daje rozwiązanie.

Widzimy, że algorytm buduje tylko rozwiązania w poprawnej formie. Pozostaje nam pokazać, że ma własność stopu.
Wystarczy zauważyć, że licznik ($a$) dąży do 0.
Ponieważ $a$ i $b$ są naturalne, wystarczy pokazać, że $a$ maleje.

$$
	a_{i} > a_{i + 1}
$$
$$
	x > x\ceil[\bigg]{\frac{y}{x}} - y
$$
$ x \neq 0 $, więc możemy podzielić przez $x$:
$$
	1 > \ceil[\bigg]{\frac{y}{x}} - \frac{y}{x}
$$

Zauważmy, że $\forall x \in \mathbb{R}$ $\left(\ceil{x} - x\right) \in [0, 1)$, co kończy dowód.

\item Czy zawsze jest to rozwiązanie optymalne (tj. o najmniejszej liczbie składników)?

Nie. Dla $\frac{13}{40}$ rozwiązanie optymalne ma 2 składniki, a zachłanne 3.

$$
	\frac{13}{40} =
    \frac{1}{5} +
    \frac{1}{8} =
    \frac{1}{4} +
    \frac{1}{14} +
    \frac{1}{280}
$$


\end{enumerate}

\end{document}