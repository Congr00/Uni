\documentclass[11pt, wide]{article}
    
    \usepackage[utf8]{inputenc}
    \usepackage[OT4]{polski}
    
    \usepackage{graphicx}
    \usepackage{caption}
    \usepackage{subcaption}
    \usepackage{epstopdf}
    \usepackage{amsmath, amssymb, amsfonts, amsthm, mathtools}
    
    \usepackage{hyperref}
    \usepackage{url}
    \usepackage{bbm}
    \usepackage{comment}
    \usepackage{makecell}
    
    \date{Wrocław, \today}
    \title{\LARGE\textbf{Pracownia z analizy numerycznej}\\Sprawozdanie do zadania \textbf{P2.8}\\
    Prowadzący pracownię: Paweł Woźny}
    \author{Łukasz Klasiński}


    \newtheorem{tw}{Twierdzenie}
    \newtheorem{alg}{Algorytm}

    \begin{document}
    \maketitle
    \thispagestyle{empty}
    \section{Wstęp}
    Problem przybliżania funkcji może być rozwiązany między innymi dzięki
    użyciu interpolacji wielomianowej. Głównym celem niniejszego sprawozdania
    będzie testowanie kilku algorytmów do obliczania wielomianu \textbf{Lagrang'a}, 
    którego postać wygląda następująco:
    \\
    \\
    Wielomian $L_n \in \Pi _n$, spełniający dla danych parami różnych liczb $x_0, x_1, \ldots x_n$ 
    oraz wartości funkcji $f$ w tych punktach, taki że 
    \\\begin{center}
        $L_n(x_k) = f(x_k), (k = 0,1,\ldots,n)$
    \end{center}
    można zapisać w postaci
    \begin{equation}\label{Lagrange}
        L_n(x) = \sum_{i=0}^{n} \lambda_i f(x_i) \prod_{j=0,j\neq i}^{n}(x - x_j)
    \end{equation}
    gdzie
    \begin{equation}
        \lambda_i = \prod_{j=0,j\neg i}^{n}\frac{1}{(t_i - t_j)}
    \end{equation}
    Zbadana zostanie stabilność różnych algorytmów na wybór węzłów oraz jaki generują błąd 
    dla danej ich ilości.
    
    Wszystkie testy numeryczne przeprowadzono przy użyciu języka programowania \textbf{Julia v.0.6.1} w trybie 
    podwójnej (\textbf{Double}) precyzji, czyli 64 bitowej dokładności.
\section{Postać Barycentryczna}
Weźmy równanie (1) i oznaczmy $l_i$ jako $\lambda_i *\prod_{j\neq i}^{n} = \prod_{j\neq i}^{n}\frac{x - x_j}{x_i - x_j} $ z (2)
\\ Zauważmy, że licznik $l_i$ może zostać zapisany jako równość
\begin{equation*}
    l(x) = (x - x_0)(x - x_1)\ldots(x - x_n)
\end{equation*} 
dzielony przez $x - x_i$. Wtedy $\lambda_i = 1/l'(x_i)$, więc $l_i$ można zapisać jako
\begin{center}
    $l_i(x) = l(x)\frac{\lambda_i}{(x - x_i)}$
\end{center}
Zauważmy, że składniki sumy (1) zawierają $l(x)$, który nie zależy od $i$. Dlatego go
wyciągnąć przed sumę, aby otrzymać
\begin{equation}
    L_n(x) = l(x)\sum_{i=0}^{n}\frac{\lambda}{x - x_i}f(x_i)
\end{equation}
Teraz załózmy, że interpolujemy funkcję stałą $ = 1$. Wtedy wstawiając to do (3),
otrzymamy równość                           
\begin{equation*}
    1 = \sum_{i=0}^{n}l_i(x) = l(x)\sum_{i=0}^{n}\frac{\lambda_i}{(x - x_i)}
\end{equation*}
Dzieląc to przez (3), otrzymamy \textsf{barycentryczną formułę} wielomianu (1)
\begin{equation}
    L_n(x) = \frac{\sum_{i=0}^{n}\frac{\lambda}{x - x_i}f(x_i)}{\sum_{i=0}^{n}\frac{\lambda}{x - x_i}}
\end{equation}
dla $\lambda$ takiej samej jak (2). Dalej, przy testowaniu będziemy go nazywać 
\textsf{wielomianem barycentrycznym}.
\\
Zauważmy, że dzięki takiej postaci $\lambda_i$ nie korzysta ze zmiennej $x$.
Dzięki temu dla zadanych węzłów wystarczy raz ją wyliczyć i przy wyznaczaniu wartości
wielomianu używać tej stałem bez ponawiania obliczeń. Ponadto jeśli 
dodamy nowe wężły, to w celu wyliczenia nowej wartości $\lambda_i$, wystarczy
przemnożyć ją przez odpowiedni iloraz $\frac{1}{t_i - t_j}$.
\subsection{Węzły Chebyszewa}
Taka postać wielomianu ma również inną własność. Jeśli zaaplikujemy
do niego węzły Chebyszewa pierwszego rodzaju:
\begin{equation}
x_i = cos\frac{(2i + 1)\pi}{2n + 2}, \hspace*{1cm} (i = 0, \ldots n)
\end{equation}
to wzór na poszczególne wartości $\lambda_i$ z (4) znacząco się upraszcza.
\\
Z własności $\lambda_i = \frac{1}{l'(x)}$, o której była mowa wcześniej, możemy
uzyskać ogólną formułę na $w_i$.
Po zaaplikowaniu do niej węzłów Chebyszewa i usunięciu czynników 
niezależnych od $i$ otrzymamy
\begin{equation}
    \lambda_i = (-1)^j sin\frac{(2j + 1)\pi}{2n + 2}
\end{equation}
Podobnie można postępować w przypadku węzłów Chebyszewa innego rodzaju.
\section{Algorytm Wernera}
Przytoczmy wielomian w postaci Newtona. Wyraża się on wzorem
\begin{equation}
    P_n(x) = \sum_{i=0}^{n}a_i p_i(x)
\end{equation}
gdzie $p_i$ jest wielomianem węzłowym
\begin{equation*}
    p_i(x) = (x - x_0)(x - x_1)\ldots(x - x_{i-1})
\end{equation*}
natomiast współczynniki $a_i$ obliczamy za pomocą ilorazu różnicowego
\begin{equation*}
    a_i = f[x_0,x_1\ldots x_i]
\end{equation*}
Główną zaletą tej formy jest mniejsza ilość operacji do wyliczenia wartości
wielomianu w porównaniu do postaci Lagrang'a. Użyjemy tą postać do wyznaczenia algorytmu
wyliczającego $\lambda_i$ w wielominie barycentrucznym z (4).
\\
Biorąc (1) oraz podstawiając za $f$ funkcję stałą $f(x) = 1$, otrzymamy
\begin{equation}
    1 = \sum_{i = 0}^n\lambda_i\prod_{j=0,j\neq i}^{n}(x - x_i)
\end{equation}
Jeśli rozważymy lewą stronę równania (8) jako wielomian w formie Newtona, to mamy
\begin{equation*}
    P_n(x) = \sum_{i = 0}^n\lambda_i\prod_{j=0,j\neq i}^{n}(x - x_i)
\end{equation*}
Możemy teraz rozwiązać ten wielomian metodą Newtona.
Po przekształceniach tego równania otrzymujemy
\begin{equation*}
    a_k = \sum_{i=0}^{k}\lambda_i\prod_{j=k+1}^{n}(x_i - x_j),\hspace*{1cm} (k = 0, \ldots n)
\end{equation*}
Zatem problem ogranicza się do rozwiązania układów równań:
\begin{equation}
    \left(\begin{array}{c}
        a_0\\
        a_1\\
        a_2\\
        \vdots\\
        a_{n-1}\\
        a_n
    \end{array}\right)
    \left(\begin{array}{ccccc}
        \prod_{j=1}^{n}(x_0 - x_j) & 0 & 0 & \ldots & 0\\
        \prod_{j=2}^{n}(x_0 - x_j) & \prod_{j=2}^{n}(x_1 - x_j) & 0 & \ldots & 0\\
        \prod_{j=3}^{n}(x_0 - x_j) & \prod_{j=3}^{n}(x_1 - x_j) & \prod_{j=3}^{n}(x_2 - x_j) & \ldots & 0\\
        \vdots & \vdots & \vdots & \vdots & \vdots\\
        x_0 - x_n & x_1 - x_n & x_2 - x_n & \ldots & 0\\
        1 & 1 & 1 & \ldots & 1
    \end{array}\right)
    \left(\begin{array}{c}
        \lambda_0\\
        \lambda_1\\
        \lambda_2\\
        \vdots\\
        \lambda_{n-1}\\
        \lambda_n
    \end{array}\right)
\end{equation} 
Może on być rozwiązany za pomocą metody eliminacji Gauss'a, który zastosujemy w 
innej kolejności niż zwykle:
\begin{enumerate}
    \item Podzielmy pierwsze równanie z (9) przez $x_0 - x_1$, oznaczmy
    \begin{equation*}
        a_0^{(1)} := a_0/(x_0 - x_1)
    \end{equation*}
    \item Odejmijmy pierwsze równianie od drugiego, oznaczmy
    \begin{equation*}
        a_1^{(1)} := a_1 - a_0^{(1)}
    \end{equation*}
    \item Podzielmy pierwsze równianie przez $x_0 - x_2$, oznaczmy
    \begin{equation*}
        a_0^{(2)} := a_0^{(1)}/(x_0 - x_2)
    \end{equation*}
    \item Odejmijmy pierwsze równianie od trzeciego, oznaczmy
    \begin{equation*}
        a_2^{(1)} := a_2 - a_0^{(2)}
    \end{equation*}
    \item Podzielmy drugie równanie przez $x_1 - x_2$, oznaczmy
    \begin{equation*}
        a_1^{(2)} := a_1^{(1)}/(x_1 - x_2)
    \end{equation*}
    \item Odejmijmy drugie równanie od trzeciego, oznaczmy
    \begin{equation*}
        a_2^{(2)} := a_2^{(1)} - a_1^{(2)} 
    \end{equation*}
\end{enumerate}
Kontynując w ten sposób otrzymujemy algorytmu   
\begin{equation}
\begin{aligned}
        &a_k^{(0)} := a_k, \hspace*{1cm} (k = 0,\ldots, n)\\
        &\left.\begin{aligned}
                &a_k^{i} := a_k^{(i-1)}/(x_k - x_i)\\
                &a_i^{(k+1)} := a_i^{(k)} - a_k^{(i)}
              \end{aligned}
        \rigth\}
        \qquad k = (0, \ldots, i-1), i = (1, \ldots, n)\\
        &\lambda_i := a_i^{(n)}, \hspace*{1cm} (i = 0, \ldots,n)
\end{aligned}
\end{equation}
Po zastosowaniu go do (8) dostajemy algorytm na wydajne obliczanie 
wartości $\lambda_i$ z (4) 
\begin{equation}
    \begin{aligned}
            &a_0^{(0)} := 0, a_k^{(0)} := 0, \hspace*{1cm} (k = 1,\ldots, n)\\
            &\left.\begin{aligned}
                    &a_k^{i} := a_k^{(i-1)}/(x_k - x_i)\\
                    &a_i^{(k+1)} := a_i^{(k)} - a_k^{(i)}
                  \end{aligned}
            \rigth\}
            \qquad k = (0, \ldots, i-1), i = (1, \ldots, n)\\
            &\lambda_i := a_i^{(n)}, \hspace*{1cm} (i = 0, \ldots,n)
    \end{aligned}
\end{equation}
Zatem otrzymaliśmy algorytm Wernera, który pozwala na szybsze obliczenie
wagi barycentrycznej niż w przypadku standardowego wyrażenia. Dzięki temu wykonujemy mniejsza
operacji i otrzymujemy mniejszy błąd przez utratę cyfr znaczących.
\section{Testy}

\section{Podsumowanie}
\begin{thebibliography}{9}
    \itemsep2pt
    \bibitem{RI} C. Schneider and W. Werner, Some New Aspects of Rational Interpolation 1986
    \bibitem{BLI} J. Berrut and L. N. Trefethen, Baricentric Lagrange Interpolation 2004
    \bibitem{LVN} W. Werner, Polynomial Interpolation: Lagrange versus Newton 1984
\end{thebibliography}    
\end{document}


